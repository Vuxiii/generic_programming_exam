\documentclass[a4paper,12pt]{report}

% Packages
\usepackage[utf8]{inputenc}
\usepackage{titling}
\usepackage{fancyhdr}
\usepackage{graphicx}
\usepackage{parskip}
\usepackage{amsmath}
\usepackage{amssymb}
\usepackage{xspace}
\usepackage{subfiles}

% Algorithms
%\usepackage{algorithm}
%\usepackage[noend]{algorithmic}
\usepackage{color}
\usepackage{listings}
\definecolor{mygray}{rgb}{0.4,0.4,0.4}
\definecolor{mygreen}{rgb}{1,0.8,0.6}
\definecolor{myorange}{rgb}{1.0,0.4,0}
\definecolor{darkpink}{rgb}{0.91, 0.33, 0.5}
\definecolor{warmyelloworange}{rgb}{0.8, 0.6, 0.2}
\lstset{
language=c++,
basicstyle=\footnotesize\sffamily\color{black},
commentstyle=\color{mygray},
frame=single,
numbers=left,
numbersep=5pt,
numberstyle=\tiny\color{mygray},
keywordstyle=\color{warmyelloworange},
showspaces=false,
showstringspaces=false,
stringstyle=\color{myorange},
tabsize=2,
breaklines=true,
captionpos=b
}
% Tabular
%\usepackage{multirow}
%\usepackage{rotating}

% Figures
%\usepackage{epic,eepic}

% Consider periods after theorems and "Proof" and consider not forcing Roman.
% Theorems
\newtheorem{xdefinition}{Definition}
\newtheorem{xobservation}{Observation}
\newtheorem{xtheorem}{Theorem}
\newtheorem{xlemma}{Lemma}
\newtheorem{xproposition}{Proposition}
\newtheorem{xcorollary}{Corollary}
\newenvironment{definition}{\begin{xdefinition}\rm}%
{\hspace*{\fill}\raisebox{-1pt}{\boldmath$\Box$}\end{xdefinition}}
\newenvironment{observation}{\begin{xobservation}\rm}%
{\hspace*{\fill}\raisebox{-1pt}{\boldmath$\Box$}\end{xobservation}}
\newenvironment{theorem}{\begin{xtheorem}\rm}{\end{xtheorem}}
\newenvironment{lemma}{\begin{xlemma}\rm}{\end{xlemma}}
\newenvironment{proposition}{\begin{xproposition}\rm}{\end{xproposition}}
\newenvironment{corollary}{\begin{xcorollary}\rm}{\end{xcorollary}}
\newenvironment{proof}{\begin{trivlist}\item[]{\bf Proof }}%
{\hspace*{\fill}\raisebox{-1pt}{\boldmath$\Box$}\end{trivlist}}

% Example commands
\newcommand{\OPT}{\ensuremath{\operatorname{\textsc{Opt}}}\xspace}
\newcommand{\MIN}[1]{\min\left\{#1\right\}}
\newcommand{\CEIL}[1]{\left\lceil#1\right\rceil}
\newcommand{\FLOOR}[1]{\left\lfloor#1\right\rfloor}
\newcommand{\SIZE}[1]{|#1|}
\newcommand{\SET}[1]{\left\{#1\right\}}
\newcommand{\SETOF}[2]{\SET{#1 \mid #2}}
\newcommand{\BIGSET}[1]{\left\{#1\right\}}
\newcommand{\BIGSETOF}[2]{\BIGSET{#1 \mid #2}}
\newcommand{\NAT}{\ensuremath{\mathbb{N}}}
\newcommand{\HRule}{\rule{\linewidth}{0.5mm}}

%%%%%%%%%%%%%%%%%%%%%%%%%%%%%%%%%%%%%%%%%%%%%%%%%%%%%%%%%%%%%%%%%%%%%%%%%%%

\begin{document}
		
\begin{titlepage}

\center % Center everything on the page

\textsc{\large Introduction to Generic Programming}\\[0.5cm]

\HRule\\[0.4cm]

{\huge\bfseries Generic Graph Library}\\[0.4cm]

\HRule\\[1.5cm]

\begin{minipage}{0.47\textwidth}
        \begin{flushleft}
                \large
                \textit{Author}\\
                William Juhl \\ 
                wijuh20@student.sdu.dk
        \end{flushleft}
\end{minipage}
\vfill\vfill

{\large June 18, 2023} % At the very end, insert date for turning in, but leave it like this during the advising

\vfill\vfill
\includegraphics[width=0.4\textwidth]{SDU_BLACK_RGB.png}\\[1cm]

\vfill

\end{titlepage}

\pagenumbering{roman} 


\newpage
\tableofcontents

\newpage
\pagenumbering{arabic} 
\setcounter{page}{1}

\chapter{Introduction}
\subfile{chapters/introduction.tex}

\chapter{Design}
\subfile{chapters/design.tex}

\chapter{Implementation}
\subfile{chapters/implementation.tex}

\chapter{Topological Sort}
\subfile{chapters/topo.tex}

\chapter{Conclusion}
\subfile{chapters/conclusion.tex}

\end{document}
